% SPDX-License-Identifier: Apache-2.0
% Copyright (c) Contributors to the PhysLight Project.

%\newacronym{ddye}{D$_{\text{dye}}$}{donor dye, ex. Alexa 488}
%\newacronym[description={\glslink{r0}{F\"{o}rster distance}}]{R0}{$R_{0}$}{F\"{o}rster distance}
%\newglossaryentry{r0}{name=\glslink{R0}{\ensuremath{R_{0}}},text=F\"{o}rster distance,description={F\"{o}rster distance, where 50\% ...}, sort=R}
%\newglossaryentry{kdeac}{name=\glslink{R0}{\ensuremath{k_{DEAC}}},text=$k_{DEAC}$, description={is the rate of deactivation from ... and emission)}, sort=k}

\newacronym[longplural=Frames per Second]{FPS}{fps}{Frame per Second}

\newacronym{MKS}{mks}{meter-kilogram-second}

\newacronym[description={The \textsl{International System of Units}
(Syst\`eme International d'Unit\'es) is a coherent system of
units of measurement (based on the \gls{MKS} system) built around seven base
units: \textit{kelvin} $[\unit{\kelvin}]$ (temperature), \textsl{second}
$[\unit{\second}]$ (time), \textsl{meter} $[\unit{\meter}]$ (length),
\textsl{kilogram} $[\unit{\kilogram}]$ (mass), \textsl{candela} $[\unit{\candela}]$
(luminous intensity), \textsl{mole} $[\unit{\mole}]$ (amount of substance) and
\textsl{ampere} $[\unit{\ampere}]$ (electric current)}]{si}{si}{International
System of Units}

\newacronym[description={Commission Internationale de l'\'Eclairage 
(International Commission on Illumination) is the international standards body that regulates over quantities related to 
illumination}]{CIE}{cie}{International
Commission on Illumination}

\newacronym{XYZ}{xyz}{CIE XYZ color space}

\newacronym{CRI}{cri}{Color Rendering Index}

\newacronym{CCT}{cct}{Correlated Color Temperature}

\newacronym{sRGB}{srgb}{sRGB color space}
\newacronym{RGB}{rgb}{RGB color space of generic primaries}

\newacronym{IBL}{ibl}{Image based lighting}

\newacronym{IES}{ies}{Illuminating Engineering Society of North America}
\newacronym{IESNA}{iesna}{Illuminating Engineering Society of North America}

\newacronym{RIB}{rib}{RenderMan Interface Bytestream}
\newacronym{RSL}{rsl}{RenderMan Shading Language}

\newacronym{LED}{led}{Light emitting diode}




\newglossaryentry{aperture}
{
	name=aperture,
	description={In a camera system, the opening in the lens that admits the light.
		The aperture is normally measured in \unit{\milli\meter}: this indicates that 
		the aperture shape is well approximated by a circle, and the amount is this 
		circle's diameter}
}

\newglossaryentry{brightness}
{
  name=brightness,
  description={The visual perception caused by the luminance emitted or
    reflected by an object. It is a subjective property of the object being
    observed and it is used in this document in its loose, plain spoken
    English sense. Other terms are used to designate specific properties}
}

\newglossaryentry{digital scene}
{
	name={digital scene},
	description={In the process of making computer-generated images, a scene is the group of
		objects to be rendered, including their shapes and material descriptions. 
		The word comes from the mental model of a real-world photographer that is using a 
		camera to take a picture of a scene}
}

\newglossaryentry{filmback}
{
	name=filmback,
	description={The area in a camera that would be covered by photosensitive film
		onto which an image is to be made. In modern camera the filmback is a photosensitive
		sensor}
}

\newglossaryentry{footage}
{
	name=footage,
	description={
		In cinema photography, a segment of motion picture material. The name derives from
	    the days of physical film: at 24 frames per second, being the standard frame rate for
    	35\unit{\milli\meter} physical film, one second of film is exactly 1.5ft.
    	Today the term is used to indicate motion picture content, be that on physical
        media or stored in digital files}
}

\newglossaryentry{illuminant}
{
	name=illuminant,
	see={digital scene},
	description={
		A source of light, meant to focus on the part of a light-emitting object that
	    specifically emits the light, as opposed to other elements of it. All together
        these constitute a \textsl{light fixture}}
}

\newglossaryentry{luminous efficacy}
{
	name={luminous efficacy},
	description={The ability of a light source to emit visible light, normally measured in
		\unit{\lumen\per\watt}}
}

\newglossaryentry{photometry}
{
	name={photometry},
	see={radiometry},
	description={The science of measurement of the strength of visible light, 
		through which are defined units that capture its perceived brightness 
		to the human eye. In photometry, light has \textsl{luminous power} 
		measured in \unit{\lumen}}
}


\newglossaryentry{pipeline}
{
	name={pipeline},
	description={The set of processes and software tools used for the production of 
		digital images, especially in digital movie-making. The word is used to draw
		attention to the large number of connections that must be put in place to 
		connect the various major tools in a production workflow, where many kinds
		of files flow together and get combined to produce the result images}
}

\newglossaryentry{radiance}
{
	name=radiance,
	description={The fundamental elemental quantity for light transport}
}

\newglossaryentry{radiometry}
{
	name={radiometry},
	see={photometry},
	description={The science of measurement of the strength of electromagnetic radiation,
		measuring its total energy, or in some cases the energy corresponding to a part of
		its spectrum. In radiometry, visible light has \textsl{radiant power},
		measured in \unit{\watt}}
}

\newglossaryentry{sensor}
{
	name=sensor,
	see={filmback},
	description={
		In a camera, the device installed at the filmback onto which the image is formed}
}

\newglossaryentry{tilt-shift lens}
{
	name={tilt-shift lens},
	description={In a camera, a lens built in such a manner that its barrel can be \emph{shifted},
		that is moved parallel to its axis, and \emph{tilted}, that is moved so make the lens axis 
		lay at an angle with respect to the filmback. These lenses are sometimes called
		\emph{perspective-control lenses}}
}


