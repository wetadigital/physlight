% SPDX-License-Identifier: Apache-2.0
% Copyright (c) Contributors to the PhysLight Project.

\chapter{Handling color}\label{ch:color}

\section{Spectral basis representation}
\label{sec:spec_basis}

In contrast to traditional RGB renderers, many contemporary renderers implement 
spectral light transport: they represent color values as functions of wavelength
internally, instead of relying purely on tristimulus values.
This allows effects such
as dispersion at dielectric boundaries to be reproduced with much higher fidelity.

Since input data is often provided to the renderer in some \gls{RGB} space,
a spectral lifting procedure is required. Such a procedure takes
tristimulus color values as an input and produces a spectrum that
corresponds to the same color.

The conversion from a spectrum $S(\lambda)$ to tristimulus values 
$(X, Y, Y)$ amounts to a projection onto a basis $(\bar x, \bar y, \bar z)(\lambda)$:
\begin{equation}
    \begin{pmatrix}X\\Y\\Z\end{pmatrix}
        = \int_\Lambda S(\lambda)\; 
    \begin{pmatrix}
        \bar x(\lambda)\\
        \bar y(\lambda)\\
        \bar z(\lambda)
    \end{pmatrix}
    \d\lambda
    \label{eqn:color_matching}
\end{equation}
While this operation is unique in the sense that there exists exactly one
value $(X,Y,Z)$ for any given spectrum $S(\lambda)$ and any particular basis
$(\bar x,\bar y, \bar z)$, the inverse operation is not. This phenomenon is called
\emph{metamerism}: spectra that are metameric to each other correspond to the
same tristimulus color value with respect to the given basis.

As a simple example, consider the family of box spectra
\begin{equation*}
    S_i(\lambda) = \begin{cases}
        i / 200, &\lambda \in [ (500-100/i)\unit{\nano\meter}, (500+100/i)\unit{\nano\meter}],\\
        0, &\text{otherwise}
    \end{cases}
\end{equation*}
for $i \geq 1 $ and the sensitivity function
\begin{equation*}
    \bar x(\lambda) = 1.
\end{equation*}
There are infinitely many such $S_i$, but for each of them,
\begin{equation*}
    \int_{380\unit{\nano\meter}}^{780\unit{\nano\meter}} S_i(\lambda) \bar x(\lambda)\d\lambda
    = 
    \int_{(500-100/i)\unit{\nano\meter}}^{(500+100/i)\unit{\nano\meter}} S(\lambda)\d\lambda
    = \frac{i}{200} \cdot \frac{200}{i} = 1,
\end{equation*}
and so these spectra are metamers with respect to $\bar x$.

This ambiguity is the main problem that needs to be solved in spectral
lifting procedures: out of the infinitely many spectra that, in general,
correspond to any given tristimulus value, one needs to be selected.

One possible selection criterion is spectrum smoothness, because naturally occurring
reflectance spectra are generally smooth functions of wavelength~\citep{maloney86}. 
The widely
used technique by~\citet{smits99} therefore attempts to find spectra that
maximize a specific smoothness criterion in an expensive precomputation.
The precomputation is run for seven colors: the sRGB primaries 
Red$(\lambda)$, Green$(\lambda)$, and Blue$(\lambda)$, 
their complements
Cyan$(\lambda)$, Magenta$(\lambda)$, and Yellow$(\lambda)$, and White$(\lambda)$. 
Smits uses a run-time interpolation scheme
that, for any given input color, attempts to use as much as possible 
of the smoothest available spectrum by first mixing in the white spectrum, 
then one of the complement color spectra, and only then a primary spectrum.

As an example, consider the input color $c_0=(0.9, 0.4, 0.1)_\text{sRGB}$:
\begin{align*}
    \min(c_{0,r}, c_{0,g}, c_{0,b}) &= 0.1\\
    c_1 &= c_0 - 0.1\cdot(1,1,1)_\text{sRGB} = (0.7, 0.3, 0)_\text{sRGB}\\
    \min(c_{1,r}, c_{1,g}) &= 0.3\\
    c_2 &= c_1 - 0.3\cdot(1,1,0)_\text{sRGB} = (0.4, 0.0, 0)_\text{sRGB}
\end{align*}
The result of this scheme is the interpolated spectrum
\begin{equation}
    S(\lambda) = 0.1\cdot \text{White}(\lambda)
               + 0.3\cdot \text{Yellow}(\lambda)
               + 0.4\cdot \text{Red}(\lambda),
\end{equation}
which corresponds to the input color since \cref{eqn:color_matching} is linear.

This technique is bound to the sRGB color space, and the sparse sampling with only
seven precomputed spectra can lead to poor results if a wider-gamut color space is
used, instead~\citep{meng2015}: saturated primary colors correspond to narrow
band spectra, which in turn produce mixtures of narrow band spectra in Smits' lifting
procedure. This can limit the maximum attainable reflectance if per-wavelength energy conservation
is required~\citep{schroedinger19}, and can be detrimental in rendering systems based
on random or quasi-random Monte Carlo integration.

\citet{meng2015} generalize the method using a much denser sampling of the xy chromaticity
space, decoupling the spectral lifting method from the input color space.

Other authors have argued that a simple smoothness criterion can be problematic
in scenes with dominant indirect illumination: 
General smoothness simply ensures that the spectrum
will be reasonably well-behaved under reflection as even narrow band incident lighting is
reflected by a wide band reflectance spectrum. However, in the presence of multiple
reflections, intricate color shifts may occur in natural materials.
\citet{otsu2017} therefore attempt to extract, for any given input color, a metameric spectrum 
a principle component analysis on measured spectral reflectance data.

This approach might benefit from domain-specific measured data sets: For example, there might
be a set of measured spectra for leaves that is used when an artist wants to recreate the
appearance of leaves.

% Naturally other means of injecting spectral data in the renderer are possible,
% such as interpreting a texture as points in a
% low-dimensional linear subspace of the space of spectral distribution, such as a PCA
% decomposition of some space of reflectances. The basis could be passed to the
% renderer both via the header of the texture itself, or some other appropriate mean.
% The linearity of the subspace would mean that the normal machinery for texture
% filtering and mipmapping would be applicable without change to such textures of coefficients.



