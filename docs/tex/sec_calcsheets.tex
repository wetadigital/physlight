\chapter{Calculation sheets}\label{ch:calcsheets}

This section will list examples for the required calculations when implementing
support for \physLight\ in a rendering pipeline.

\section{Imaging}\label{ch:calc_calibration}

We will use the following camera Inputs for our calibration scenario described
in Chapter~\ref{ch:imaging}:
\begin{itemize}
\item Film speed $S = \SI{100}{\iso}$
\item Focal length $f = \SI{24}{\milli\meter}$
\item Focus distance $o = \SI{1}{\meter}$
\item Iris aperture $N = \SI{8}{\fnumber}$
\item Exposure time $t = \SI{1/60}{\second}$
\item Sensor width/height $\SI{27.7 x 14.6}{\milli\meter}$ (RED Epic Mysterium-X)
\item Image resolution $\SI{5120 x 2700}{\pixel}$ (RED Epic Mysterium-X 5K)
\item Pixel area $A_{pxl} \simeq \SI{29.25}{\square\micro\meter}$
\item Calibration constant $C = 312.5$
\end{itemize}

\paragraph{Incident Light Meter Exposure}
Compute illuminance $E_v$ at $P$ given imaging parameters:
\begin{align*}
E_v& = C\frac{N^2}{t S}\\
&= 312.5\frac{8\cdot 8 \cdot 60}{100} = \SI{12000}{\lux}
\end{align*}

\paragraph{Reflected Luminance}
Compute the reflected luminance with $18\%$ reflective Lambertian plane:
\begin{align*}
L_v& = E_v\cdot \frac{\rho}{\pi} \\
&= \num{12480} \cdot \frac{0.18}{\pi}
\simeq \SI{715.0}{\nit}
\end{align*}

\paragraph{Entrance pupil distance}

Compute the distance $a$ from the entrance pupil to the filmback given:
\begin{itemize}
\item Focal length $f = \SI{24}{\milli\meter}$
\item Focus distance $o = \SI{1}{\meter}$
\end{itemize}

The lens equation in our case has the form
\begin{displaymath}
\frac1f = \frac1a + \frac1{o - a}
\end{displaymath}
solving for $a$ gives us a quadratic equation $a^2 - a o + f o = 0$ for which we
choose the solution closest to the filmback:
\begin{displaymath}
a = \frac{o-\sqrt{o^2 - 4fo}}2
\end{displaymath}

Substituting our data into this expression yields
\begin{displaymath}
a = \frac{1-\sqrt{1 - 4 \cdot 0.024}}2 \;\si{\meter} \simeq \SI{24.6}{\milli\meter}
\end{displaymath}

\ifomit
\subsection{Luminous energy}

\begin{itemize}
\item Sensitivity $S = \SI{100}{\iso}$
\item Focal length $f = \SI{24}{\milli\meter}$
\item Iris aperture $N = \SI{8}{\fnumber}$
\item Exposure time $t = \SI{1/60}{\second}$
\item Calibration constant $C = 312.5$
\end{itemize}

That gives us:
\begin{align*}
\theta = \arctan\frac{f}{2N(o-a)}, && L_v = \frac{\rho}{\pi}\cdot
C\frac{N^2}{\Delta t S},
\end{align*}

\begin{displaymath}
E_v = 312.5\frac{8\cdot 8 \cdot 60}{100} = \SI{12000}{\lux}
\end{displaymath}


With this $k_i$ can be calculated as:
\begin{displaymath}
k_i = \frac{Y}{\cdots}
\end{displaymath}

\begin{figure}
% SPDX-License-Identifier: Apache-2.0
% Copyright (c) Contributors to the PhysLight Project.
\begin{tikzpicture}[line cap=round,line join=round,>=triangle 45,x=1.0cm,y=1.0cm]
%\draw [color=eqeqeq,, xstep=1.0cm,ystep=1.0cm] (-4,2) grid (6,7);
%\draw[->,color=black] (0,2) -- (0,7);
%\foreach \y in {2,3,4,5,6}
%\draw[shift={(0,\y)},color=black] (2pt,0pt) -- (-2pt,0pt) node[left] {\footnotesize $\y$};
\clip(-4,2) rectangle (6,7);
\draw [shift={(0,4.5)},color=qqwuqq,fill=qqwuqq,fill opacity=0.1] (0,0) -- (-14.11:0.5) arc (-14.11:-9.46:0.5) -- cycle;
\draw [shift={(0,4.5)},color=qqwuqq,fill=qqwuqq,fill opacity=0.1] (0,0) -- (-18.44:0.5) arc (-18.44:-14.11:0.5) -- cycle;
\draw [shift={(5.44,3.59)},color=qqwuqq,fill=qqwuqq,fill opacity=0.1] (0,0) -- (170.54:0.3) arc (170.54:224.74:0.3) -- cycle;
\draw [shift={(4.76,2.91)},color=qqwuqq,fill=qqwuqq,fill opacity=0.1] (0,0) -- (44.74:0.3) arc (44.74:161.56:0.3) -- cycle;
\draw [shift={(5.07,3.23)},color=qqwuqq,fill=qqwuqq,fill opacity=0.1] (0,0) -- (134.74:0.3) arc (134.74:165.89:0.3) -- cycle;
\draw[color=qqwuqq,fill=qqwuqq,fill opacity=0.1] (5.2,3.36) -- (5.07,3.49) -- (4.94,3.36) -- (5.07,3.23) -- cycle;
\draw [line width=1pt] (-3,6)-- (-3,3);
\draw [line width=1pt] (0,6)-- (0,5);
\draw [line width=1pt] (0,4)-- (0,3);
\draw [domain=-4:6] plot(\x,{(-3.51--1.93*\x)/1.95});
\draw [domain=-3.0:6.0] plot(\x,{(--13.5-0.75*\x)/3});
\draw [dash pattern=on 1pt off 1pt,domain=-3.0:6.0] plot(\x,{(--13.5-1*\x)/3});
\draw [dash pattern=on 1pt off 1pt,domain=-3.0:6.0] plot(\x,{(--13.5-0.5*\x)/3});
\draw [->] (5.07,3.23)-- (4.47,3.83);
\begin{scriptsize}
\draw[color=uuuuuu] (0.13,4.75) node {$F$};
\draw[color=uuuuuu] (-3.33,5.17) node {$pxl$}
;
\draw[color=uuuuuu] (5.55,3.33) node {$P_2$};
\draw[color=uuuuuu] (5.15,3.0) node {$P$};
\draw[color=uuuuuu] (4.85,2.7) node {$P_1$};
\draw[color=uuuuuu] (4.5 ,4.0) node {$N$};

\draw[color=qqwuqq] (.85,4.5) node {$\alpha_2$};
\draw[color=qqwuqq] (.85,4.0) node {$\alpha_1$};

\draw[color=qqwuqq] (5.4,4.0) node {$\beta_2$};
\draw[color=qqwuqq] (4.5,2.85) node {$\beta_1$};
\draw[color=qqwuqq] (4.4,3.55) node {$\theta$};
\end{scriptsize}
\end{tikzpicture}

\caption{Footprint at $P$ through perspective projection}
\end{figure}

Compute luminous energy $Q_v$ at a certain pixel given the exposure parameters:

\begin{itemize}
\item Sensitivity $S = \SI{100}{\iso}$
\item Aperture distance $a = \SI{25}{\milli\meter}$
\item Iris aperture $N = \SI{8}{\fnumber}$
\item Exposure time $t = \SI{1/60}{\second}$
\item Focus distance $o = \SI{1}{\meter}$
\item Illuminance $E = \SI{12480}{\lux}$
\item Pixel area $A_{pxl} = \SI{18.5}{\square\micro\meter}$
\end{itemize}

\begin{figure}

\begin{tikzpicture}[line cap=round,line join=round,>=triangle 45,x=1.0cm,y=1.0cm]
%\draw [color=cqcqcq,dash pattern=on 1pt off 1pt, xstep=1.0cm,ystep=1.0cm] (-1,-1) grid (9,4);
%\draw[->,color=black] (-1,0) -- (9,0);
%\foreach \x in {-1,1,2,3,4,5,6,7,8}
%\draw[shift={(\x,0)},color=black] (0pt,2pt) -- (0pt,-2pt) node[below] {\footnotesize $\x$};
%\draw[->,color=black] (0,-1) -- (0,4);
%\foreach \y in {-1,1,2,3}
%\draw[shift={(0,\y)},color=black] (2pt,0pt) -- (-2pt,0pt) node[left] {\footnotesize $\y$};
%\draw[color=black] (0pt,-10pt) node[right] {\footnotesize $0$};
\clip(-1,-1) rectangle (9,4);
\draw [shift={(0,0)},color=qqwuqq,fill=qqwuqq,fill opacity=0.1] (0,0) -- (0.27:0.6) arc (0.27:22.39:0.6) -- cycle;
\draw [shift={(4.28,0.02)},color=qqwuqq,fill=qqwuqq,fill opacity=0.1] (0,0) -- (131.12:0.5) arc (131.12:180.27:0.5) -- cycle;
\draw[color=qqwuqq,fill=qqwuqq,fill opacity=0.1] (4.6,0.3) -- (4.32,0.62) -- (4,0.34) -- (4.28,0.02) -- cycle;
\draw [shift={(8.06,3.32)},color=qqwuqq,fill=qqwuqq,fill opacity=0.1] (0,0) -- (-157.61:0.6) arc (-157.61:-138.88:0.6) -- cycle;
\draw[color=qqwuqq,fill=qqwuqq,fill opacity=0.1] (3.8,1.18) -- (4.13,1.32) -- (3.99,1.64) -- (3.67,1.51) -- cycle;
\draw [domain=-1:9] plot(\x,{(-0--0.02*\x)/4.28});
\draw [domain=-1:9] plot(\x,{(-0--3.32*\x)/8.06});
\draw [line width=1pt] (4.28,0.02)-- (8.06,3.32);
\draw [shift={(4.28,0.02)},color=qqwuqq] (131.12:0.6) arc (131.12:180.27:0.6);
\draw [shift={(4.28,0.02)},color=qqwuqq] (131.12:0.5) arc (131.12:180.27:0.5);
\draw [->] (4.28,0.02)-- (3.71,0.67);
\draw [dash pattern=on 1pt off 1pt,domain=-1:9] plot(\x,{(-14.05--3.3*\x)/3.78});
\draw [dash pattern=on 1pt off 1pt on 2pt off 4pt] (4.28,0.02)-- (3.67,1.51);
\draw [shift={(8.06,3.32)},color=qqwuqq] (-157.61:0.6) arc (-157.61:-138.88:0.6);
\draw[color=qqwuqq] (7.6,3.04) -- (7.5,2.97);
\begin{scriptsize}
\draw[color=uuuuuu] (0,-0.2) node {$F$};
\draw[color=uuuuuu] (4.38,-0.2) node {$P$};
\draw[color=uuuuuu] (8.3,3.15) node {$P_2$};
\draw[color=qqwuqq] (1.0,0.23) node {$\alpha_2$};
\draw[color=uuuuuu] (3.55,0.85) node {$N$};
\draw[color=qqwuqq] (3.65,0.3) node {$\theta$};
\draw[color=uuuuuu] (3.52,1.75) node {$H_2$};
\draw[color=qqwuqq] (7.3,2.85) node {$\beta_2$};
\end{scriptsize}
\end{tikzpicture} 

\caption{Pixel footprint computation}
\label{fig:opposite-angle}
\end{figure}

A pixel receives energy from an area $A_P$ around $P$ which is computed as follows: let $\alpha_1$ and $\alpha_2$ be the angles spanned by a ray $r$ through $P$ and the center of the pixel, and a rays through the border of the same pixel along $x$ (in raster coordinates) in the positive and negative direction, these last rays intersect  the plane at $P$ in two points $P_1$ and $P_2$ respectively, at angles $\beta_1$ and $\beta_2$.

From Figure~\ref{fig:opposite-angle} it can be seen how to compute the length of segment $PP_2$ from the length of segment $PF$ and the angles $\alpha_2$ and $\theta$. It is immediately seen that:
\begin{displaymath}
\overline{PH_2} = \overline{FP}\sin\alpha_2 \qquad \overline{PP_2} = \frac{\overline{PH_2}}{\sin\beta_2}
\end{displaymath}
it is also easily seen that $\beta_2 = \pi - \alpha_2 - (\pi/2 + \theta)$ which implies $\sin\beta_2 = \cos(\theta-\alpha_2)$. Substituting and reasoning similarly for segment $PP_1$ we obtain
\begin{displaymath}
\overline{P_1P_2} = \overline{PF}
	\left(\frac{\sin\alpha_1}{\cos(\theta+\alpha_1)} +
	      \frac{\sin\alpha_2}{\cos(\theta-\alpha_2)}\right)
\end{displaymath}

In the case in which $\alpha_1 = \alpha_2 = \alpha/2$ we obtain with some manipulation
\begin{displaymath}
\overline{P_1P_2} = \overline{PF}
	\frac{2\cos\theta\sin\alpha}{1+\cos2\theta\cos\alpha}
\end{displaymath}


This lets us approximate the area $A_P$ of the footprint of a pixel projected around $P$ as $\overline{P_1P2}^2$, under the approximation that a pixel subtends equal angles $\alpha$ in the two raster directions, so that
\begin{displaymath}
A_P \simeq \overline{PF}^2
    \left( \frac{2\cos\theta\sin\alpha}{1+\cos2\theta\cos\alpha} \right)^2
\end{displaymath}

To compute $Q_v$ we need to integrate luminance over $A_P$,
the solid angle $\Delta\omega$ subtended by the aperture as seen from $P$
and the exposure time.

To compute $\Delta\omega$ we consider the angle $\gamma$ formed by $PF$ and the
optical axis of the system. This lets us compute
\begin{displaymath}
\Delta\omega =  \pi \left(\frac f{2N}\right)^2 \frac{\cos\gamma}{\overline{PF}^2}
\end{displaymath}
which takes us to our expression for $Q_v$:
\begin{align*}
Q_v & = L_o(\omega_o)\cdot A_P \cdot\Delta\omega \cdot t \\
    & = L_o(\omega_o) \cdot \overline{PF}^2
    \left( \frac{2\cos\theta\sin\alpha}{1+\cos2\theta\cos\alpha} \right)^2 \cdot \pi \left(\frac f{2N}\right)^2 \frac {\cos\gamma}{\overline{PF}^2} \cdot t \\
    & = L_o(\omega_o) \cdot \left( \frac{2\cos\theta\sin\alpha}{1+\cos2\theta\cos\alpha} \right)^2 \cdot \pi \left(\frac f{2N}\right)^2 \cos\gamma \cdot t
\end{align*}

If the surface is Lambertian at $P$, it illuminates equally in all directions:

\begin{displaymath}
L_v = \frac\rho\pi E_v = \frac{0.18}{3.14159}12\,480 = \SI{715.0}{\nit}
\end{displaymath}
substituting the parameters we obtain:
\begin{align*}
Q_v & = L_o(\omega_o) \cdot \left( \frac{2\cos\theta\sin\alpha}{1+\cos2\theta\cos\alpha} \right)^2 \cdot \pi \left(\frac f{2N}\right)^2 \cos\gamma \cdot t \\
    & = 715 \cdot 18.5\cdot10^{-12} \cdot \frac{975\cdot10^{-3}}{25\cdot10^{-3}} 3.14159 \left(\frac{24}{8}\right)^2\frac{1}{975^2\cdot 10^{-6}} \frac{1}{60} \\
    & = \SI{2.5574e-7}{\talbot}
\end{align*}

\fi

\paragraph{Solid Angle of the Aperture}
Calculate $\Omega_{obj}$ the solid angle of the aperture seen from point in object space:
\begin{align*}
\Omega_p^{obj} &\simeq \frac{\pi r^2}{(o-a)^2}\\
&= \frac{f^2}{4N^2(o-a)^2}\\
&\simeq  \SI{0.74296e-5}{\steradian}
\end{align*}

\paragraph{Luminous energy from Y value}

Given a pixel $Y$ value we can calculate the luminous energy density in \si{\lux} for this pixel by
\begin{align*}
Y_v &= \frac{Y\cdot K_m}{ k_i\cdot S \cdot \Delta t} \\
&= \frac{Y\cdot C}{ 4\cdot S \cdot \Delta t} \left(\frac{f}{a}\right)^2
\qquad [\si{\lux}]
\end{align*}
For $Y=0.18$ and the given input specifications we get:
\begin{align*}
Y_v &= \frac{Y\cdot C}{ 4\cdot S \cdot \Delta t} \left(\frac{f}{a}\right)^2 \\
&= \frac{0.18\cdot 312.5}{ 4\cdot 100 \cdot 1/60} \left(\frac{0.024}{0.0246}\right)^2 \;\si{\lux} \\
&= \SI{8.030934}{\lux}
\end{align*}

\section{Emission constant}

\subsection{Area Light}
Given following input parameters for an area light:
\begin{itemize}
\item luminous power $\Phi_v=\SI{1000}{\lumen}$
\item area $|A|= \SI{2 x 2}{\meter} = \SI{4}{\square\meter}$
\item Lambertian angular distribution $D = \frac1{\pi}$
\item emitter: black body radiator at $T=\SI{6500}{\kelvin}$
\end{itemize}
we want to compute the light constant $k_e$. From Chapter~\ref{ch:specification} we can  have:
\begin{displaymath}
 k_e = \frac{\Phi_v}{K_m \cdot |A| \cdot \|D\| \cdot \langle \|T_{col}\|, \|col\cdot\hat{L}\|_{\bar y} \rangle }
\end{displaymath}

\paragraph{Area Light Without Texturing} Without a tint function this reduces to
\begin{displaymath}
k_e = \frac{\Phi_v}{K_m \cdot |A| \cdot \|D\| \cdot \|\hat{L}\|_{\bar y} }
\end{displaymath}
with $\|\hat{L}\|_{\bar y} = \|B_T\|_{\bar y}$ which can be precomputed with the
closed-form expression in Chapter~\ref{ch:illuminants}. This gives us
\begin{align*}
\|\hat{L}\|_{\bar y} &= \|B_{6500}\|_{\bar y} \simeq 2826.5 \\
k_e &= \frac{1000}{683 \cdot 4 \cdot 2826.5 } \simeq 1.295 \cdot 10^{-4}
\end{align*}

\paragraph{Area Light With Texturing}
Using the same input parameters but adding a texture as a tint function,
we have to calculate the following integrals:
\begin{displaymath}
\|T_{col}\| = \int T_{col}(x) \d x
\end{displaymath}

Under the assumption that the texture coordinate mapping onto the light has no area
distortion, this integral reduces to average of the \gls{RGB} values of the
texture map scaled by the light area $A$:
\begin{align*}
\|T_{rgb}\| = A \overline{T}_{rgb}
\end{align*}
where $\overline{T}_{rgb}$ is the average of the the pixel values in the
texture,
a value that can be precomputed and stored in the header of the texture.

The reduced luminous power vector $\|col\cdot \hat L\|_{\bar y}$ can also
be precomputed for the black body radiator at the given temperature.
\begin{displaymath}
\|col\cdot \hat L\|_{\bar y} = \int col(\lambda) \cdot B_T(\lambda) \cdot \bar y
(\lambda) \d\lambda
\end{displaymath}
where $col$ are the basis functions of the space used to convert the texture's
\gls{RGB} data into its spectral representation.
This gives us
\begin{align*}
 k_e &= \frac{\Phi_v}{K_m \cdot |A| \cdot \|D\| \cdot \langle \overline{T}_{rgb}, \|col\cdot\hat{L}\|_{\bar y} \rangle } \\
&= \frac{1000}{683 \cdot 4 \cdot \langle \overline{T}_{rgb}, \|col\cdot\hat{L}\|_{\bar y} \rangle  }
\end{align*}
where $\langle \bar T, \phi \rangle = \bar T_r|\phi_r| + \bar T_g|\phi_g| + \bar T_b|\phi_b| $
in the simple case of a simplistic three basis functions lifting to spectral.

Other, more plausible ways of lifting \gls{RGB} data to the spectral domain
are possible, as explained in Section~\ref{sec:spec_basis}.

For example, if using the algorithm proposed by Brian Smits in~\cite{smits99},
the expression $\langle \bar T, \phi \rangle$ this would be computed as follows:
the various $col$ functions will correspond to the seven basis functions of the
Smits basis $w(λ), c(λ), m(λ), y(λ), r(λ), g(λ), b(λ)$, yielding seven integrals
$\phi_w, \phi_c, \phi_m, \phi_y, \phi_r, \phi_g, \phi_b$ corresponding to
$\phi_w =  \|w\cdot\hat{L}\|_{\bar y}$ and so on.

The the algorithm produces a vector $\bar T \in \R^7$ in which at most three
components are non-zero as follows: the first entry $T_0$ is the smallest component
of $\overline{T_{rgb}}$, say that this was the green component. Then the basis
vector from the secondaries is chosen (cyan, magenta and yellow: $c(\lambda), m(\lambda), y(\lambda)$)
and its corresponding value (one of $T_1, T_2, T_3$) is set to the difference
between the smallest value and the middle value. Lastly, the primary corresponding
to the largest value is chosen (red, green, and blue: $r(\lambda), g(\lambda), b(\lambda)$)
and its corresponding component (one of $T_4, T_5, T_6$) is set to the difference
between the largest and middle values.

For example if $\overline{T_{rgb}} = (0.3, 0.1, 0.7)$ then $T_0$ is set to $0.1$,
then magenta is the complementary, so $T_2$ is set to $0.3-0.1 = 0.2$, then
blue is the primary, so $T_6$ is set to $0.7 - 0.3 = 0.4$, yielding
$\overline{T} = (0.1, 0, 0.2, 0, 0, 0, 0.4)$. An implementation in code of
this algorithm is available in Section~\ref{sec:implementation:texturemaps}.

\section{Surface interaction}

Compute the exitant radiance $L_{o\lambda}(\omega_o, \lambda)$ from a point
$P_s$ on a surface of normal $N$, given the incident radiance
$L_{i\lambda}(\omega_i, \lambda)$ and the object’s BRDF $f(\omega_i, \omega_o, \lambda)$.
Let us assume that the surface at $P_s$ is made of Lambertian material of reflectance $\rho=0.18$, i.e., $f(\omega_i, \omega_o, \lambda) = \frac \rho \pi$. 

\begin{displaymath}
\begin{split}
    L_{o,v}(\omega_o) &= K_m \int_\lambda \int_\Omega L_{i,\lambda}(\omega_i, \lambda) f(\omega_i, \omega_o, \lambda) \d \omega^{\bot} \bar y(\lambda) \d\lambda \\
    &= K_m \int_\lambda \int_0^\pi \int_0^{\frac \pi 2}  L_{i,\lambda}(\omega_i, \lambda)\;f(\omega_i, \omega_o, \lambda)\, \bar y(\lambda)\, \cos\theta \sin\theta \d\theta \d\phi \d\lambda \\
    &= \int_0^\pi \int_0^{\frac \pi 2}  L_{i,v}(\omega_i)\;f(\omega_i, \omega_o)\cos\theta \sin\theta \d\theta \d\phi
\end{split}
\end{displaymath}

Let's say that the light is a portion of a sphere of radius $r = \SI{1}{\meter}$, centered
around our surface point $P_s$, extending in $\theta$ from $\frac\pi6$ to $\frac\pi3$ and in
$\phi$ from $0$ to $\frac\pi2$, and having Lambertian emission towards our
surface of $L_{i,v} = \SI{300}{\lumen\per\square\meter\per\steradian}$.
In this configuration we obtain:

\begin{align*}
L_{o,v} &= \int_{\theta = \frac\pi 6}^\frac\pi 3 \int _{\phi = 0}^\frac\pi
2 L_{i,v} \frac\rho\pi \cos\theta \sin\theta \d\theta \d\phi \\
             &= 300\cdot \frac{0.18}\pi \int_{\theta = \frac\pi 6}^\frac\pi 3 \int _{\phi = 0}^\frac\pi 2 \cos\theta \sin\theta \d\theta \d\phi \\
             &= 54 \left.\frac{-\cos^2\theta}{2}\right|_{\frac\pi 6}^\frac\pi 3 \\
             &= \SI{13.5}{\lumen\per\square\meter\per\steradian}
\end{align*}

\section{Light to sensor interaction}

Given camera parameters $S=100$, $N=5.6$, $\Delta t = \frac 1 {60}$
and an infinitesimally small light source oriented towards a
Lambertian plane parallel to the light source with distance
$d = \SI{0.4}{\meter}$. If the luminance $Y(p)$ at the pixel corresponding
to the center of the projection $p$ equals the albedo $\rho$ of the plane,
what is the luminous power $\Phi_v$ of the light source?
\begin{figure}[t]
    \centering
    \def\svgwidth{0.9\linewidth}
    \import{figures/}{emitter_surface.pdf_tex}
    \caption{Integration over area light.}
\end{figure}

According to Equation~\eqref{eqn:imaging_Ev} the illuminance at the plane must be 
\begin{displaymath}
  E_v = \frac{C\;N^2}{\Delta t\; S} = 5880\; \si{\lux}.
\end{displaymath}
Say the light source has a cosine powered distribution
$L_v^{\uparrow,light}(\omega) = \alpha \cos^n \theta'$, where
$\theta'$ denotes the angle between emission direction $\omega$ and
the light normal, then due to
\begin{displaymath}
  M_v = \int_\Omega L_v^{\uparrow,light}(\omega) \; \d \omega^\bot \stackrel{\eqref{eq:powered_cosine}}= \alpha \frac{2\pi}{n+2}
\end{displaymath}
and $M_v = \Phi_v / |A|$
we have
\begin{equation}\label{eq:light_exitance}
  L_v^{\uparrow,light}(\omega) = \frac{\Phi_v}{|A|} \frac{n+2}{2\pi} \cos^n \theta'.
\end{equation}
The illuminance at $p$ is
\begin{multline*}
  E_v = \int_\Omega L_v^{\downarrow,obj}(\omega) \d\omega^{\bot} = \int_A L_v^{\uparrow,light}(x\to p) \frac{\cos \theta\; \cos \theta'}{|x-p|^2} \d x \\
  \stackrel{\eqref{eq:light_exitance}} =  \frac{\Phi_v}{|A|} \frac{n+2}{2\pi} \int_A  \frac{\cos^{n+1} \theta\; \cos \theta'}{|x-p|^2} \d x
  \approx |A| \frac{ \Phi_v }{|A|\; d^2} \frac{n+2}{2\pi} \d x  \stackrel{n=0} \approx \frac{\Phi_v}{\pi\;d^2}
\end{multline*}
where the approximation holds if the light source is sufficiently small, hence the cosines are $\approx 1$ and assuming a Lambertian emission ($n=0$).
Putting this together, it follows 
\begin{displaymath}
  \Phi_v = \pi \; 0.4^2 \si{\square\meter}\cdot 5880\; \si{\lux} = 2955.61\; \si{\lumen}.
\end{displaymath}

Simple geometric setups like these can be reproduced in a render and are hence useful for validation purposes.
